\documentclass[a4paper,12pt]{article}

\usepackage{prettylatex}
\usepackage{titlepage}
\usepackage{boiboites}
\usepackage{pgfplots}

\top{Université de Technologie de Belfort-Montbéliard}{}
\title{Cours d'IN41}{Chapitre 7 -- Signaux et systèmes discrets}
\author{}
\date{Semestre de printemps 2016}

\newboxedtheorem[boxcolor=orange, background={rgb:white,20;green,2;black,1}, titlebackground={rgb:white,15;green,5;black,3},
titleboxcolor = black]{defi}{Définition}{cmpDefi}

\begin{document}

\maketitlepage

\tableofcontents
\pagebreak

\section{Signaux discrets}

$\chi$ : séquence de nombres dans laquelle le $n^{eme}$ nombre est $x(n)$ \\
Notation : $\chi = \{ x(n) \}$ avec $-\infty < n < +\infty$

\subsection{Signaux classiques}

\subsubsection{Impulsion unité}

\[ \delta(n) = \begin{cases}
    1 & \text{ si } n = 0 \\
    0 & \text{ sinon}
\end{cases} \]

\subsubsection{Saut unité}

\[ \Gamma(n) = \begin{cases}
    1 & \text{ si } n >= 0 \\
    0 & \text{ sinon}
\end{cases} = \sum_{k=0}^{+\infty} \delta(n-k) \]
\[ \delta(n) = \Gamma(n) - \Gamma(n-1) \]

\subsubsection{Exponentielle numérique}

\[ x(n) = R^n \Gamma(n) \]

Si $-1 < R < 1 \implies$ exponentielle décroissante \\
Si $|R| > 1 \implies$ exponentielle croissante

\subsubsection{Sinusoïde}

\[ x(n) = \cos(n\omega_0 + e) \text{ et } \omega_0 = 2\pi f_0 t_e \]

\subsubsection{Phaseur de pulsations $\omega_0$}

\[ x(n) = e^{in\omega_0} \]

\subsection{Propriétés des signaux discrets}

Énergie totale : $E(\infty) = \sum_{-\infty}^{+\infty} ||x(n)||^2$ \\
Puissance moyenne : $P_n = \lim_{N \to +\infty} \frac{1}{N} \sum_{-N/2}^{N/2} |x(n)|^2$
$P$-périodique : $x(n) = x(n+P) \forall n$

\section{Systèmes numériques}

{\Large Schéma 7}

Notation : $y(n) = \T{x(n)}$

Classification :

\begin{itemize}
    \item statique : $y(n)$ ne dépend que de $x(n)$ au même instant ;
    \item dynamique : $y(n)$ est une fonction de $x(n)$ aux instants antérieurs ou égaux à $n$ et/ou des échantillons de sortie.
\end{itemize}

Exemple :
\[ y(n) = b_0 x(n) + b_1 x(n-1) + a_1 y(n-1) + a_2 y(n-2) \]

Schéma fonctionnel ou diagramme fontionnel : illustration graphique des opérations effectuées sur le signal d'entrée ainsi que les connexions les reliant (addition, multiplication, décalage avant et arrière)

\[ y(n) = \frac{1}{2} ((x_1(n) + x_1(n-1))x_2(n)) - \frac{1}{4} y(n-1) \]

{\Large Illustration 8}

Interconnexion des systèmes :

En cascade : $x_1(n)$

\end{document}
