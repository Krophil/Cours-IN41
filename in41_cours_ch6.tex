\documentclass[a4paper,12pt]{article}

\usepackage{prettylatex}
\usepackage{titlepage}
\usepackage{boiboites}
\usepackage{pgfplots}

\top{Université de Technologie de Belfort-Montbéliard}{}
\title{Cours d'IN41}{Chapitre 6 -- Filtrage analogique}
\author{}
\date{Semestre de printemps 2016}

\newboxedtheorem[boxcolor=orange, background={rgb:white,20;green,2;black,1}, titlebackground={rgb:white,15;green,5;black,3},
titleboxcolor = black]{defi}{Définition}{cmpDefi}

\begin{document}

\maketitlepage

\tableofcontents
\pagebreak

\section{Filtres idéaux}

\subsection{Filtre passe bas}

Laisse passer uniquement les fréquences $f < f_c$, avec $f_c$ la fréquence de coupure.

\[ \text{Gain : } \begin{cases}
    1 & \forall f < f_c \\
    0 & \text{ sinon}
\end{cases} \]

{\Large Graphique 1}

\subsection{Filtre passe haut}

Laisse passer uniquement les fréquences $f >= f_p$ avec $f_p$ la fréquence passante.

\[ \text{Gain : } \begin{cases}
    1 & \forall f >= f_c \\
    0 & \text{ sinon}
\end{cases} \]

{\Large Graphique 2}

\subsection{Filtre passe bande}

Laisse passer les fréquences $f \in [ f_p , f_c [$.

{\Large Graphique 3}

\subsection{Filtre coupe bande}

Coupe une plage de fréquence $f \in [ f_c , f_p [$.

{\Large Graphique 4}

\section{Filtres réels}

Un filtre est défini par sa réponse impulsionnelle.

\[ s(t) = e(t) \otimes h(t) \text{ avec } \begin{cases}
    \TF{h(t)} = H(f) \\
    \TL{h(t)} = H(p)
\end{cases} \]
\[ s(t) = \int_{-\infty}^{\infty} e(\nu) h(t-\nu) \d \nu \]

{\Large Graphique 5}

\[ S(f) = H(f) E(f) \]

Si l'entrée est un signal causal ($x(t) = 0 \forall t < 0$) alors $s(t) = \int_0^t e(\nu) h(t-\nu) \d \nu$ (car la réponse impulsionnelle est causale) $\implies$ la fonction de transfert est obligatoirement complexe.

\[ H(f) = ||H(f)|| e^{i \times \textrm{arg}(H(f))} \]
\[ S(f) = E(f) H(f) = E(f) ||H(f)|| e^{i \times \textrm{arg}(H(f))} \]

Cas d'un filtre passe-bas idéal :
\[ H(f) = \textrm{RECT}(f) \]
\[ h(t) = \int_{-\infty}^{+\infty} H(f) e^{i2\pi ft} \d f \]
\[ = \int_{-f_c}^{f_c} e^{i2\pi ft} \d f = [ \frac{e^{i2\pi ft}}{i2\pi t} ]_{-f_c}^{f_c} \]
\[ = \frac{e^{i2\pi f_ct} - e^{-i2\pi f_ct}}{i2\pi t} = \frac{\sin(2\pi f_c t)}{2\pi f_c t} 2f_c = 2f_c \sinc(2\pi f_c t) \]

$h(t)$ obtenue n'est pas causale $\implies$ on obtient un filtre.
Le passe bas idéal n'est pas réalisable.

Les filtres réels présentent des imperfections en fonction de leurs applications \\
$\rightarrow$ transition progressive entre la bande passante et la bande coupée \\
$\rightarrow$ irrégularité du gain dans la bande passante (ondulation) \\
$\rightarrow$ irrégularité du gain dans la bande coupée (ondulation) \\
$\rightarrow$ irrégularité du temps de propagation

\subsection{Filtre passe bas}

Bande passante : $\omega_p = \omega_p - 0$ \\
Bande de transition : $\omega_a - \omega_p$ \\
Bande coupée : $+\infty - \omega_a$ \\
Sélectivité : $s = \frac{\omega_p}{\omega_a}$

\subsection{Filtre passe haut}

Bande passante : $+\infty - \omega_p$ \\
Bande de transition : $\omega_p - \omega_a$ \\
Bande coupée : $\omega_a = \omega_a - 0$ \\
Sélectivité : $s = \frac{\omega_a}{\omega_p}$

\subsection{Filtre passe-bande}

Bande passante : $\omega_{2p} - \omega_{1p}$ \\
Bandes de transition : $\omega_{2a} - \omega_{2p}$ et $\omega_{1p} - \omega_{1a}$ \\
Bandes atténuées : $+\infty - \omega_{2a}$ et $\omega_{1a} - 0$ \\
Pulsation centrale : $\omega_0 = \sqrt{\omega_{2p} - \omega_{1p}}$ \\
Sélectivité : $s = \frac{\omega_{2p} - \omega_{1p}}{\omega_{2a} = \omega_{1a}}$

\subsection{Filtre coupe bande}

Bandes passante : $+\infty - \omega_{2p}$ et $\omega_{1p} - 0$ \\
Bandes de transition : $\omega_{2p} - \omega_{2a}$ et $\omega_{1a} - \omega_{1p}$ \\
Bande coupée : $\omega_{2a} - \omega_{1a}$ \\
Sélectivité : $s = \frac{\omega_{2a} - \omega_{1a}}{\omega_{2p} = \omega_{1p}}$

\subsection{Filtre de Butterworth}

Défini par la fonction de transfert

\[ H(f)^2 = \frac{1}{1+(\frac{f}{f_c})^{2n}} \text{ avec } \begin{cases}
    n & \text{ : ordre du filtre} \\
    f_c & \text{ : fréquence de coupure}
\end{cases} \]

Propriétés :

\begin{itemize}
    \item Réponse plate dans la bande passante et dans la bande coupée \\
    $\forall n \in \mathbb{N}^* ||H(f_c)||^2 = 1/2 \implies G(f_c) = 20 \mathrm{log}(||H(f_c)||) = 3 \mathrm{dB}$
    \item Atténuation asymptotique de $-20n \mathrm{dB}$ / décade
    \item L'ordre du filtre et de la fonction de transfert est déterminé à patir du gabarit
\end{itemize}

Table de Butterworth :

\begin{tabular}{|l|l|}
    \hline
    $n$ & Polynôme $P_n(p)$ \\
    \hline
    $1$ & $p+1$ \\
    $2$ & $p^2 + \sqrt{2}p + 1$ \\
    $3$ & $(p+1)(p^2 + p + 1)$ \\
    $4$ & $(p^2 + 0.765p + 1)(p^2 + 1.84p + 1)$ \\
    \hline
\end{tabular}

\end{document}
